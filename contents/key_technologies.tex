% !TeX encoding = UTF-8
% !TeX root = ../main.tex

%% ------------------------------------------------------------------------
%% Copyright (C) 2021-2023 SJTUG
%% 
%% SJTUBeamer Example Document by SJTUG
%% 
%% SJTUBeamer Example Document is licensed under a
%% Creative Commons Attribution-NonCommercial-ShareAlike 4.0 International License.
%% 
%% You should have received a copy of the license along with this
%% work. If not, see <http://creativecommons.org/licenses/by-nc-sa/4.0/>.
%% -----------------------------------------------------------------------

\section{关键技术与实现难点}

\begin{frame}
  \frametitle{1. 技术调研与需求分析}
  
  \begin{block}{调研与分析内容}
    \begin{itemize}
      \item \textbf{现有解决方案调研}:长时间、持续的监测提高农作物管理的精确度和及时性。
      \item \textbf{技术需求分析}:无人机在远距离和高频次的物流任务中能够持续运作,提高物流效率。
      \item \textbf{技术挑战识别}:识别和评估可能遇到的技术挑战,如能量传输效率、充电稳定性、环境适应性等。
  \end{itemize}
  \end{block}
  \end{frame}
  
  \begin{frame}
  \frametitle{2. 硬件设计}
  
  \begin{block}{无线充电板设计}
      \begin{itemize}
          \item 设计与无人机电池兼容的无线充电板,确保高效能量传输。
          \item 考虑电磁兼容性和安全性,避免对无人机其他电子设备的干扰。
      \end{itemize}
  \end{block}
  
  \begin{block}{对接机构设计}
      \begin{itemize}
          \item 设计机械对接结构,确保无人机能够稳定地降落在充电板上。
          \item 设计电气连接系统,确保充电过程的稳定性和安全性。
      \end{itemize}
  \end{block}
  
  \end{frame}
  
  \begin{frame}
  \frametitle{3. 算法开发}
  
  \begin{block}{自动导航与定位算法}
      \begin{itemize}
          \item 开发无人机的自动导航和定位算法,实现精准的飞行路径规划。
          \item 确保无人机能够在不同环境下准确定位,避免障碍物。
      \end{itemize}
  \end{block}
  
  \begin{block}{自动对接算法}
      \begin{itemize}
          \item 开发自动对接算法,包括障碍物识别、降落点识别和精确降落控制。
          \item 确保无人机能够安全、准确地降落在充电板上,实现自动充电。
      \end{itemize}
  \end{block}
  
  \end{frame}
  
  \begin{frame}
  \frametitle{4. 系统集成}
  
  \begin{block}{硬件与软件集成}
      \begin{itemize}
          \item 将无线充电板、对接机构、导航与对接算法集成到一个完整的系统中。
          \item 确保硬件和软件之间的协调工作,实现系统的整体功能。
      \end{itemize}
  \end{block}
  
  \begin{block}{初步测试与调试}
      \begin{itemize}
          \item 进行系统的初步测试和调试,发现并解决潜在的问题。
          \item 优化系统性能,确保各组件的稳定运行。
      \end{itemize}
  \end{block}
  
  \end{frame}
  
  \begin{frame}
  \frametitle{5. 实地测试}
  
  \begin{block}{受控环境测试}
      \begin{itemize}
          \item 在受控环境中对无人机的自动充电系统进行测试,验证其功能和性能。
          \item 收集测试数据,分析系统的工作状态和充电效果。
      \end{itemize}
  \end{block}
  
  \begin{block}{数据分析与优化}
      \begin{itemize}
          \item 根据测试数据进行分析,找出系统的不足和改进点。
          \item 进一步优化系统,提高充电效率和系统稳定性。
      \end{itemize}
  \end{block}
  
  \end{frame}
