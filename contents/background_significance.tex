% !TeX encoding = UTF-8
% !TeX root = ../main.tex

%% ------------------------------------------------------------------------
%% Copyright (C) 2021-2023 SJTUG
%% 
%% SJTUBeamer Example Document by SJTUG
%% 
%% SJTUBeamer Example Document is licensed under a
%% Creative Commons Attribution-NonCommercial-ShareAlike 4.0 International License.
%% 
%% You should have received a copy of the license along with this
%% work. If not, see <http://creativecommons.org/licenses/by-nc-sa/4.0/>.
%% -----------------------------------------------------------------------

\section{选题背景及意义}

\begin{frame}[fragile]
  \frametitle{无人机应用的广泛性与重要性}

  无人机(Unmanned Aerial Vehicles, UAVs)因其卓越的灵活性和成本效益,已经在众多领域得到了广泛应用,包括但不限于:

  \begin{itemize}
    \item \textbf{农业监测}
    \begin{itemize}
      \item 高效监测农作物生长情况
      \item 评估病虫害
      \item 优化农业资源配置
    \end{itemize}

    \item \textbf{物流配送}
    \begin{itemize}
      \item 快速配送
      \item 特殊环境下的物资运输
    \end{itemize}

    \item \textbf{灾害救援}
    \begin{itemize}
      \item 提供现场情况
      \item 支持救援工作
    \end{itemize}
  \end{itemize}
  
\end{frame}

\begin{frame}[fragile]
  \frametitle{无人机续航能力的限制}
  尽管无人机应用广泛,但其续航能力受到电池容量和能量密度的限制。这一问题显著限制了无人机执行长时间和复杂任务的能力,特别是在以下情况下:
  \begin{itemize}
    \item \textbf{长时间巡航任务:}如大面积农田监测和森林火灾预警。
    \item \textbf{远距离物流配送:}如偏远地区的药品和急需物资配送。
    \item \textbf{连续环境监测:}如河流和湖泊的长期水质监测。
  \end{itemize}
\end{frame}

\begin{frame}[fragile]
  \frametitle{项目的创新性与应用价值}
  本项目旨在设计并实现一套无人机自动无线充电系统,其创新性和实际应用价值体现在:
  \begin{itemize}
    \item \textbf{无人机平台开发:}适合执行各种任务的四旋翼无人机
    \item \textbf{长时间巡航任务:}如大面积农田监测和森林火灾预警。
    \item \textbf{远距离物流配送:}如偏远地区的药品和急需物资配送。
    \item \textbf{连续环境监测:}如河流和湖泊的长期水质监测。
  \end{itemize}
\end{frame}

\begin{frame}
  \frametitle{项目意义}
  
  \begin{block}{1. 提高无人机的续航能力}
      \begin{itemize}
          \item \textbf{自动充电}:通过无线充电技术,无人机可以在任务过程中自动进行充电,无需更换电池,从而大幅延长其工作时间。
          \item \textbf{减少停机时间}:自动充电减少了无人机因电池耗尽而必须停机更换电池的时间,提高了工作效率。
      \end{itemize}
  \end{block}
  
  \begin{block}{2. 增强任务执行的灵活性}
      \begin{itemize}
          \item \textbf{适应多种任务需求}:无线充电技术使无人机能够在各种任务场景中灵活充电,适应长时间和复杂任务需求。
          \item \textbf{提升应急响应能力}:在紧急任务中,无人机能迅速恢复电力,持续执行任务。
      \end{itemize}
  \end{block}
\end{frame}
  
\begin{frame}
  \begin{block}{3. 推动无人机技术的智能化发展}
      \begin{itemize}
          \item \textbf{自主导航与对接}:开发无人机自动对接和精确降落算法,提高无人机的自主性和智能化水平。
          \item \textbf{系统集成与优化}:通过硬件和软件的集成与优化,实现无人机的全自动充电系统,推动无人机技术的发展。
      \end{itemize}
  \end{block}
  
  \begin{block}{4. 提升行业应用的效率和效果}
      \begin{itemize}
          \item \textbf{农业监测}:长时间、持续的监测提高农作物管理的精确度和及时性。
          \item \textbf{物流配送}:无人机在远距离和高频次的物流任务中能够持续运作,提高物流效率。
          \item \textbf{环境监测}:提供更长时间和更广范围的环境数据采集,提升环境保护和治理的效果。
      \end{itemize}
  \end{block}
\end{frame}

\begin{frame}
  \begin{block}{5. 促进新技术和新应用的开发}
      \begin{itemize}
          \item \textbf{跨领域技术融合}:结合无线充电技术和无人机技术,推动新技术的发展和应用。
          \item \textbf{引领产业创新}:为其他行业提供技术参考和创新思路,如自动驾驶车辆、智能家居设备的无线充电应用。
      \end{itemize}
  \end{block}


  \begin{block}{6. 节约资源和环保}
      \begin{itemize}
          \item \textbf{减少电池更换频率}:通过无线充电减少对电池的更换需求,延长电池使用寿命,降低资源消耗。
          \item \textbf{降低环境污染}:减少废旧电池的处理和更换,降低对环境的污染。
      \end{itemize}
  \end{block}
  
\end{frame}